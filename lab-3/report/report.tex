% Created 2019-10-13 Вс 22:38
% Intended LaTeX compiler: pdflatex
\documentclass[11pt]{article}
\usepackage[utf8]{inputenc}
\usepackage[T1]{fontenc}
\usepackage{graphicx}
\usepackage{grffile}
\usepackage{longtable}
\usepackage{wrapfig}
\usepackage{rotating}
\usepackage[normalem]{ulem}
\usepackage{amsmath}
\usepackage{textcomp}
\usepackage{amssymb}
\usepackage{capt-of}
\usepackage{hyperref}
\usepackage[T2A]{fontenc}
\usepackage[a4paper,left=3cm,top=2cm,right=1.5cm,bottom=2cm,marginparsep=7pt,marginparwidth=.6in]{geometry}
\usepackage{cmap}
\usepackage{xcolor}
\usepackage{listings}
\usepackage{polyglossia}
\setdefaultlanguage{russian} \setotherlanguage{english}
\setmainfont{Liberation Serif}
\setsansfont{Liberation Sans}
\setmonofont[Contextuals=Alternate,Ligatures={TeX}]{Fira Code Regular}
\author{Krutko Nikita / KrutNA}
\date{\today}
\title{}
\hypersetup{
 pdfauthor={Krutko Nikita / KrutNA},
 pdftitle={},
 pdfkeywords={},
 pdfsubject={},
 pdfcreator={Emacs 26.1 (Org mode 9.1.9)}, 
 pdflang={Russian}}
\begin{document}

\large
\thispagestyle{empty}
\begin{center}
\textbf{Национальный Исследовательский Университет ИТМО}\\
\textbf{Факультет Программной Инженерии и Компьютерной Техники}\\
\end{center}
\vspace{2em}
\begin{center}
\includegraphics[width=120pt]{../itmo-logo.png}
\end{center}
\LARGE
\vspace{5em}
\begin{center}
\textbf{Вариант № 14 \% 9 = 5}\\
\textbf{Лабораторная работа № 3}\\
\Large
\textbf{по дисциплине}\\
\LARGE
\textbf{\emph{'Информатика'}}\\
\end{center}
\vspace{11em}
\large
\begin{flushright}
\textbf{Выполнил:}\\
\textbf{Студент группы P3113}\\
\textbf{\emph{Крутько Никита;} : 242570}\\
\textbf{Преподаватель:}\\
\textbf{\emph{Малышева Татьяна Алексеевна}}\\
\end{flushright}
\vspace{4em}
\large
\begin{center}
\textbf{Санкт-Петербург 2019 г.}
\end{center}
\pagebreak{}
\setcounter{tocdepth}{2}
\tableofcontents
\normalsize
\section{Задание}
\label{sec:orgf09ad0d}
\subsection{Файл}
\label{sec:org559666b}
\begin{itemize}
\item Создать следующего вида исходный файл из десяти строк, содержащий в каждой строке ФИО, дату рождения, дату получения паспорта и баллы ЕГЭ по трём предметам:
\end{itemize}
\scriptsize
\begin{verbatim}
Апельсинов А.А. 08.02.2000 17.03.2014 90 100 91
Виноградов В.В. 09.03.1999 15.04.2013 67 99 98
Яблоков Я.Я. 10.04.2000 19.05.2014 94 94 94
Морковкин М.М. 11.05.1999 17.06.2013 91 82 73
\end{verbatim}

\subsection{Заданиеи}
\label{sec:orgaa3348c}
\normalsize
\begin{itemize}
\item Не используя готовые сторонние подключаемые функции для факториала, int(), bin() и т.п., написать программу на языке Python 3.x, которая бы вычисляла среднее значение баллов ЕГЭ, сортировала строки по указанной колонке в обратном порядке (от большего к меньшему) и выводила результат следующего вида (для сортировки по дате рождения):
\end{itemize}
\scriptsize
\begin{verbatim}
Морковкин М.М. | 11.05.1999 | 17.06.2013 | 91 82 73 -> 82.000000
Яблоков Я.Я. | 10.04.2000 | 19.05.2014 | 94 94 94 -> 94.000000
Виноградов В.В. | 09.03.1999 | 15.04.2013 | 67 99 98 -> 88.000000
Апельсинов А.А. | 08.02.2000 | 17.03.2014 | 90 100 91 -> 93.666667
\end{verbatim}

\section{Код Python}
\label{sec:org33f9800}
\scriptsize
\lstset{language=Python,label= ,caption= ,captionpos=b,numbers=none}
\begin{lstlisting}
#!/usr/bin/env python3
def get_avg(array):
    return sum(int(i) for i in array)/len(array)


def split(line):
    res = line.split(' ')
    return [' '.join(res[0:2]), res[2], res[3],
	    ' '.join(res[4:]), get_avg(res[4:])]


def sort(array, col=4, ascending=True):
    return sorted(array, key=lambda val: val[col], reverse=not ascending)


def prnt(line):
    print("%s -> %f" % (' | '.join(line[:4]), line[4]))


if __name__ == '__main__':
    file = "file"
    with open(file, "r") as f:
	result = list(map(lambda line: split(line.rstrip()), f.readlines()))
    result = sort(result, 1, False)
    print(result)
    map(lambda line: prnt(line), result)
\end{lstlisting}

\section{Вывод}
\label{sec:org9e0eacf}
\normalsize
Немножко кода на питоне, ничего особенного
\end{document}