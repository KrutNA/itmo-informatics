% Created 2019-09-29 Вс 12:25
% Intended LaTeX compiler: pdflatex
\documentclass[11pt]{article}
\usepackage[utf8]{inputenc}
\usepackage[T1]{fontenc}
\usepackage{graphicx}
\usepackage{grffile}
\usepackage{longtable}
\usepackage{wrapfig}
\usepackage{rotating}
\usepackage[normalem]{ulem}
\usepackage{amsmath}
\usepackage{textcomp}
\usepackage{amssymb}
\usepackage{capt-of}
\usepackage{hyperref}
\usepackage[T2A]{fontenc}
\usepackage[a4paper,left=3cm,top=2cm,right=1.5cm,bottom=2cm,marginparsep=7pt,marginparwidth=.6in]{geometry}
\usepackage{cmap}
\usepackage{xcolor}
\usepackage{listings}
\usepackage{polyglossia}
\setdefaultlanguage{russian} \setotherlanguage{english}
\setmainfont{Liberation Serif}
\setsansfont{Liberation Sans}
\setmonofont[Contextuals=Alternate,Ligatures={TeX}]{Fira Code Regular}
\author{Крутько Никита}
\date{\today}
\title{}
\hypersetup{
 pdfauthor={Крутько Никита},
 pdftitle={},
 pdfkeywords={},
 pdfsubject={},
 pdfcreator={Emacs 26.1 (Org mode 9.1.9)}, 
 pdflang={Russian}}
\begin{document}

\begin{center}
\textbf{Санкт-Петербургский Национальный Исследовательский}\\
\textbf{Университет Информационных Технологий, Механики и Оптики}\\
\textbf{Факультет Программной Инженерии и Компьютерной Техники}\\
\end{center}
\vspace{1em}
\begin{center}
\includegraphics[width=120pt]{../itmo-logo.png}
\end{center}
\LARGE
\vspace{5em}
\begin{center}
\textbf{Вариант №17}\\
\textbf{Лабораторная работа №2}\\
\Large
\textbf{по дисциплине}\\
\LARGE
\textbf{\emph{'Информатика'}}\\
\end{center}
\vspace{11em}
\large
\begin{flushright}
\textbf{Выполнил:}\\
\textbf{Студент группы P3113}\\
\textbf{\emph{Крутько Никита} : 242570}\\
\textbf{Преподаватель:}\\
\textbf{\emph{Малышева Татьяна Алексеевна}}\\
\end{flushright}
\vspace{4em}
\large
\begin{center}
\textbf{Санкт-Петербург 2019 г.}
\end{center}
\pagebreak{}
\setcounter{tocdepth}{2}
\tableofcontents
\vspace{2em}
\section{Описание}
\label{sec:org76da9de}
Дано: A = 12893; C = 13547
\section{Задание 1}
\label{sec:org33ca740}
\subsection{Описание}
\label{sec:org514c6ee}
По заданному варианту исходных данных получить набор десятичных чисел:
\subsection{Решение}
\label{sec:orgb2e538e}
\setlength{\tabcolsep}{3pt}
\begin{table}[htbp]
\caption{\label{tab:org85e00a0}
Значения X}
\centering
\begin{tabular}{|llcr|lcrl|}
\hline
 & X1 & \(A\) & 12893 & X7 & \(-X7\) & -12893 & \\
 & X2 & \(C\) & 13547 & X8 & \(-X8\) & -13547 & \\
 & X3 & \(A+C\) & 26440 & X9 & \(-X9\) & -26440 & \\
 & X4 & \(A+2C\) & 39987 & X10 & \(-X10\) & -39987 & \\
 & X5 & \(C-A\) & 654 & X11 & \(-X11\) & -654 & \\
 & X6 & \(65536-X4\) & 25549 & X12 & \(-X12\) & -25549 & \\
\hline
\end{tabular}
\end{table}

\section{Задание 2}
\label{sec:org27d3fe9}
\subsection{Описание}
\label{sec:org98d07dc}
Выполнить перевод десятичных чисел X1,…,X6 в двоичную систему счисления, получив их двоичные эквиваленты B1,…,B6 соответственно.
Не использовать при этом никакой формат представления данных, не использовать никакую разрядную сетку.

\subsection{Решение}
\label{sec:org7f990ea}
\setlength{\tabcolsep}{3pt}
\label{tab:org201f51e}
\begin{tabular}{llllclcl}
 &  &  & X1\(_{\text{10}}\) & \(\to\) & B1\(_{\text{2}}\) & = & 11001001011101\\
 &  &  & X2\(_{\text{10}}\) & \(\to\) & B2\(_{\text{2}}\) & = & 11010011101011\\
 &  &  & X3\(_{\text{10}}\) & \(\to\) & B3\(_{\text{2}}\) & = & 110011101001000\\
 &  &  & X4\(_{\text{10}}\) & \(\to\) & B4\(_{\text{2}}\) & = & 1001110000110011\\
 &  &  & X5\(_{\text{10}}\) & \(\to\) & B5\(_{\text{2}}\) & = & 1010001110\\
 &  &  & X6\(_{\text{10}}\) & \(\to\) & B6\(_{\text{2}}\) & = & 110001111001101\\
\end{tabular}

\section{Задание 3}
\label{sec:orgcbfb477}
\subsection{Описание}
\label{sec:org7f2548e}
Используя 16-разрядный двоичный формат со знаком и полученные в предыдущем пункте задания двоичные числа B1,…,B6 (т.е. при необходимости дополнить числа B1…B6 ведущими нулями и однозначно интерпретировать эти числа в 16-разрядном двоичном формате со знаком), вычислить двоичные числа B7,…,B12: B7 = -B1, B8 = -B2, B9 = -B3, B10 = -B4, B11 = -B5, B12 = -B6. Отрицательные числа представлять в дополнительном коде

\subsection{Решение}
\label{sec:org3d5c4f6}
\setlength{\tabcolsep}{3pt}
\label{tab:org68df9ab}
\begin{tabular}{llllcrcl}
 &  &  & B1\(_{\text{10}}\) & \(\to\) & B1\(_{\text{2}}\) & = & 0011001001011101\\
 &  &  & B2\(_{\text{10}}\) & \(\to\) & B2\(_{\text{2}}\) & = & 0011010011101011\\
 &  &  & B3\(_{\text{10}}\) & \(\to\) & B3\(_{\text{2}}\) & = & 0110011101001000\\
 &  &  & B4\(_{\text{10}}\) & \(\to\) & B4\(_{\text{2}}\) & = & 1001110000110011\\
 &  &  & B5\(_{\text{10}}\) & \(\to\) & B5\(_{\text{2}}\) & = & 0000001010001110\\
 &  &  & B6\(_{\text{10}}\) & \(\to\) & B6\(_{\text{2}}\) & = & 0110001111001101\\
\end{tabular}
\setlength{\tabcolsep}{3pt}
\label{tab:org972b976}
\begin{tabular}{llllcrcl}
 &  &  & B7\(_{\text{2}}\) & \(\to\) & -B1\(_{\text{2}}\) & = & 1100110110100011\\
 &  &  & B8\(_{\text{2}}\) & \(\to\) & -B2\(_{\text{2}}\) & = & 1100101100010101\\
 &  &  & B9\(_{\text{2}}\) & \(\to\) & -B3\(_{\text{2}}\) & = & 1001100010111000\\
 &  &  & B10\(_{\text{2}}\) & \(\to\) & -B4\(_{\text{2}}\) & = & 0011001111000111\\
 &  &  & B11\(_{\text{2}}\) & \(\to\) & -B5\(_{\text{2}}\) & = & 1111110101110010\\
 &  &  & B12\(_{\text{2}}\) & \(\to\) & -B6\(_{\text{2}}\) & = & 1001110000110011\\
\end{tabular}

\section{Задание 4}
\label{sec:org106862f}
\subsection{Описание}
\label{sec:org6cb83b3}
Найти область допустимых значений для данного двоичного формата.

\subsection{Решение}
\label{sec:org01418aa}

ОДЗ: -32768..32767

\section{Задание 5}
\label{sec:orgb0c7335}
\subsection{Описание}
\label{sec:orgb6953ab}
Выполнить обратный перевод двоичных чисел B1…B12 (используя 16-разрядный двоичный формат со знаком) в десятичные и прокомментировать полученные результаты.

\subsection{Решение}
\label{sec:org78d9d18}
\setlength{\tabcolsep}{3pt}
\label{tab:org5009e7a}
\begin{tabular}{llllclcll}
 &  &  & B1\(_{\text{10}}\) & \(\to\) & Н1\(_{\text{10}}\) & = & 12893 & Compares\\
 &  &  & B2\(_{\text{10}}\) & \(\to\) & Н2\(_{\text{10}}\) & = & 13547 & Compares\\
 &  &  & B3\(_{\text{10}}\) & \(\to\) & Н3\(_{\text{10}}\) & = & 26440 & Compares\\
 &  &  & B4\(_{\text{10}}\) & \(\to\) & Н4\(_{\text{10}}\) & = & -25549 & Not compares\\
 &  &  & B5\(_{\text{10}}\) & \(\to\) & Н5\(_{\text{10}}\) & = & 654 & Compares\\
 &  &  & B6\(_{\text{10}}\) & \(\to\) & Н6\(_{\text{10}}\) & = & 25549 & Compares\\
\end{tabular}
\setlength{\tabcolsep}{3pt}
\label{tab:orgaa52d04}
\begin{tabular}{llllclcll}
 &  &  & B7\(_{\text{10}}\) & \(\to\) & Н7\(_{\text{10}}\) & = & -12893 & Compares\\
 &  &  & B8\(_{\text{10}}\) & \(\to\) & Н8\(_{\text{10}}\) & = & -13547 & Compares\\
 &  &  & B9\(_{\text{10}}\) & \(\to\) & Н9\(_{\text{10}}\) & = & -26440 & Compares\\
 &  &  & B10\(_{\text{10}}\) & \(\to\) & Н10\(_{\text{10}}\) & = & 13255 & Not compares\\
 &  &  & B11\(_{\text{10}}\) & \(\to\) & Н11\(_{\text{10}}\) & = & -654 & Compares\\
 &  &  & B12\(_{\text{10}}\) & \(\to\) & Н12\(_{\text{10}}\) & = & -25549 & Compares\\
\end{tabular}

\section{Задание 6}
\label{sec:orgdce295f}
\subsection{Описание}
\label{sec:org472f3a5}
Выполнить следующие сложения двоичных чисел: B1+B2, B2+B3, B2+B7, B7+B8, B8+B9, B1+B8, B11+B3 (итого, 7 операций сложения).\\
Для представления слагаемых и результатов сложения использовать 16-разрядный двоичный формат со знаком. Результаты сложения перевести в десятичную систему счисления, сравнить с соответствующими десятичными числами (т.е. сравнить с суммой слагаемых, представленных в десятичной системе: B1 + B2 vs X1 + X2).

\subsection{Решение}
\label{sec:org9d9a981}
\subsubsection{B1 + B2}
\label{sec:org4f3d3ba}
\setlength{\tabcolsep}{3pt}
\label{tab:org5e24624}
\begin{tabular}{lllllrlllllr}
 &  &  &  & B1\(_{\text{2}}\) & 11001001011101 &  &  &  &  & X1 & 12893\\
 &  &  & + & B2\(_{\text{2}}\) & 11010011101011 &  &  &  & + & X2 & 13547\\
 &  &  &  &  & ------------------------ &  &  & = &  &  & ---------\\
 &  &  &  &  & 110011101001000 & = & 26440 &  &  &  & 26440\\
\end{tabular}

CF = 0; PF = 1; AF = 1; ZF = 0; SF = 0; OF = 0\\
При сложении двух положительных слагаемых получено положительное число. Результат выполнения операции верный и корректный, совпадает с суммой десятичных эквивалентов.

\subsubsection{B2 + B3}
\label{sec:org33f1565}
\setlength{\tabcolsep}{3pt}
\label{tab:org7d4ee9f}
\begin{tabular}{lllllrlllllr}
 &  &  &  & B2\(_{\text{2}}\) & 11010011101011 &  &  &  &  & X2 & 13547\\
 &  &  & + & B3\(_{\text{2}}\) & 110011101001000 &  &  &  & + & X3 & 26440\\
 &  &  &  &  & ------------------------ &  &  & = &  &  & ---------\\
 &  &  &  &  & 1001110000110011 & = & -25549 &  &  &  & 39987\\
\end{tabular}

CF = 0; PF = 1; AF = 1; ZF = 0; SF = 1; OF = 1\\
При сложении двух положительных слагаемых получено отрицательное число. Результат выполнения операции неверный и некорректный, не совпадает с суммой десятичных эквивалентов.

\subsubsection{B2 + B7}
\label{sec:org4d01f90}
\setlength{\tabcolsep}{3pt}
\label{tab:orgdede01d}
\begin{tabular}{lllllrlllllr}
 &  &  &  & B2\(_{\text{2}}\) & 11010011101011 &  &  &  &  & X2 & 13547\\
 &  &  & + & B7\(_{\text{2}}\) & 1100110110100011 &  &  &  & + & X7 & -12893\\
 &  &  &  &  & ------------------------ &  &  & = &  &  & ---------\\
 &  &  &  &  & 0000001010001110 & = & 654 &  &  &  & 654\\
\end{tabular}

CF = 0; PF = 1; AF = 0; ZF = 0; SF = 0; OF = 0\\
При сложении двух положительных слагаемых получено положительное число. Результат выполнения операции верный и корректный, совпадает с суммой десятичных эквивалентов.

\subsubsection{B7 + B8}
\label{sec:orga834b8e}
\setlength{\tabcolsep}{3pt}
\label{tab:org55b6420}
\begin{tabular}{lllllrlllllr}
 &  &  &  & B7\(_{\text{2}}\) & 1100110110100011 &  &  &  &  & X7 & -12893\\
 &  &  & + & B8\(_{\text{2}}\) & 1100101100010101 &  &  &  & + & X8 & -13547\\
 &  &  &  &  & ------------------------ &  &  & = &  &  & ---------\\
 &  &  &  &  & 1001100010111000 & = & -26440 &  &  &  & -26440\\
\end{tabular}

CF = 0; PF = 1; AF = 0; ZF = 0; SF = 1; OF = 0\\
При сложении двух отрицательных слагаемых получено отрицательное число. Результат выполнения операции верный и корректный, совпадает с суммой десятичных эквивалентов.

\subsubsection{B8 + B9}
\label{sec:orgfd79346}
\setlength{\tabcolsep}{3pt}
\label{tab:org5059467}
\begin{tabular}{lllllrlllllr}
 &  &  &  & B8\(_{\text{2}}\) & 1100101100010101 &  &  &  &  & X8 & -13547\\
 &  &  & + & B9\(_{\text{2}}\) & 1001100010111000 &  &  &  & + & X9 & -26440\\
 &  &  &  &  & ------------------------ &  &  & = &  &  & ---------\\
 &  &  &  &  & 0110001111001101 & = & 25549 &  &  &  & -39987\\
\end{tabular}

CF = 1; PF = 0; AF = 0; ZF = 0; SF = 0; OF = 1\\
При сложении двух отрицательных слагаемых получено положительное число. Результат выполнения операции неверный и некорректный, не совпадает с суммой десятичных эквивалентов.

\subsubsection{B1 + B8}
\label{sec:org1acdaae}
\setlength{\tabcolsep}{3pt}
\label{tab:orgcda4d71}
\begin{tabular}{lllllrlllllr}
 &  &  &  & B1\(_{\text{2}}\) & 11001001011101 &  &  &  &  & X1 & 12893\\
 &  &  & + & B9\(_{\text{2}}\) & 1001100010111000 &  &  &  & + & X9 & -26440\\
 &  &  &  &  & ------------------------ &  &  & = &  &  & ---------\\
 &  &  &  &  & 1100101100010101 & = & -13547 &  &  &  & -13547\\
\end{tabular}

CF =0; PF = 0; AF = 1; ZF = 0; SF = 1; OF = 0\\
При сложении отрицательного и положительного слагаемых получено отрицательное число. Результат выполнения операции верный и корректный, совпадает с суммой десятичных эквивалентов.

\subsubsection{B11 + B3}
\label{sec:org1bb0e74}
\setlength{\tabcolsep}{3pt}
\label{tab:orgfed2912}
\begin{tabular}{lllllrlllllr}
 &  &  &  & B11\(_{\text{2}}\) & 1111110101110010 &  &  &  &  & X11 & -654\\
 &  &  & + & B3\(_{\text{2}}\) & 110011101001000 &  &  &  & + & X3 & 26440\\
 &  &  &  &  & ------------------------ &  &  & = &  &  & ---------\\
 &  &  &  &  & 0110010010111010 & = & 25786 &  &  &  & 25786\\
\end{tabular}

CF = 1; PF = 0; AF = 0; ZF = 0; SF = 0; OF = 0\\
При сложении отрицательного и положительного слагаемых получено отрицательное число. Результат выполнения операции верный и корректный, совпадает с суммой десятичных эквивалентов.

\section{Вывод}
\label{sec:orge245702}
В ходе выполнения лабораторной работы я изучил как выполнять операции с двоичными числами на доп коде, написал прогрыммы на языке \emph{Python}, которые это делают, а также изучил флаги состояния процессора.
\end{document}